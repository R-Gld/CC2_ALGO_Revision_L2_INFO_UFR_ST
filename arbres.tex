\subsection{Généralités sur les Arbres}\label{subsec:generalites-sur-les-arbres}

\subsubsection{Définitions}
Un arbre est une structure de données hiérarchique composée de nœuds, où chaque nœud peut avoir zéro ou plusieurs enfants.
Un arbre ne contient pas de cycles, et il y a un seul chemin entre deux nœuds quelconques.

\subsubsection{Types d'arbres}
\begin{itemize}
    \item \textbf{Arbre binaire} : Chaque nœud a au plus deux enfants.
    \item \textbf{Arbre binaire complet} : Tous les niveaux, sauf éventuellement le dernier, sont entièrement remplis.
    \item \textbf{Arbre binaire équilibré} : La différence de hauteur entre les sous-arbres gauche et droit de tout nœud est au plus un.
\end{itemize}

\subsubsection{Applications des arbres}
Les arbres sont utilisés dans de nombreuses applications comme les systèmes de gestion de bases de données, l'analyse syntaxique dans les compilateurs, les algorithmes de recherche, etc.

\subsubsection{Terminologie des arbres}
\begin{itemize}
    \item \textbf{Racine} : Le nœud supérieur de l'arbre.
    \item \textbf{Feuille} : Un nœud sans enfants.
    \item \textbf{Branche} : Un nœud avec au moins un enfant.
    \item \textbf{Hauteur} : Longueur du chemin le plus long de la racine à une feuille.
\end{itemize}

\subsubsection{Exemples concrets}
Un exemple concret d'utilisation des arbres est la représentation des dossiers et fichiers dans un système d'exploitation, où chaque dossier est un nœud et les fichiers/dossiers qu'il contient sont ses enfants.

\subsection{Arbres Binaires}\label{subsec:arbres-binaires}

\subsubsection{Définitions et Propriétés}
Un arbre binaire est un type spécial d'arbre dans lequel chaque nœud a au plus deux enfants, souvent désignés comme enfant gauche et enfant droit.
Il est utilisé dans de nombreuses applications en informatique en raison de sa structure simple et efficace.

\subsubsection{Représentation en C}
\begin{lstlisting}[label={lst:imp_BT}]
struct struct Node {
    int data;
    struct struct Node* left;
    struct struct Node* right;
};
\end{lstlisting}
Cette structure en C représente un nœud d'un arbre binaire avec une valeur de données, un pointeur vers le fils gauche et un pointeur vers le fils droit.

    \subsubsection{Parcours d'arbres binaires}
Les arbres binaires peuvent être parcourus de différentes manières :

\paragraph{Parcours préfixe}
Visite d'abord la racine, puis parcours du sous-arbre gauche, suivi du sous-arbre droit.
\newline
\begin{tikzpicture}[level distance=1.5cm,
  level 1/.style={sibling distance=3cm},
  level 2/.style={sibling distance=1.5cm}]
  \node (A) {A}
    child {node (B) {B}
      child {node (D) {D}}
      child {node (E) {E}}
    }
    child {node (C) {C}
      child {node (F) {F}}
      child {node (G) {G}}
    };

  % Parcours préfixe en rouge
  \draw[->, red, -Stealth] (A) -- (B);
  \draw[->, red, -Stealth] (B) -- (D);
  \draw[->, red, -Stealth] (B) -- (E);
  \draw[->, red, -Stealth] (A) -- (C);
  \draw[->, red, -Stealth] (C) -- (F);
  \draw[->, red, -Stealth] (C) -- (G);

    \node [right=of A, yshift=-1.5cm, xshift=3.5cm] {A$\rightarrow$B$\rightarrow$D$\rightarrow$E$\rightarrow$C$\rightarrow$F$\rightarrow$G};


\end{tikzpicture}

\paragraph{Parcours infixe}
Parcours d'abord le sous-arbre gauche, visite la racine, puis parcours le sous-arbre droit.
\newline
\begin{tikzpicture}[level distance=1.5cm,
  level 1/.style={sibling distance=3cm},
  level 2/.style={sibling distance=1.5cm}]
  \node (A) {A}
    child {node (B) {B}
      child {node (D) {D}}
      child {node (E) {E}}
    }
    child {node (C) {C}
      child {node (F) {F}}
      child {node (G) {G}}
    };

  % Parcours infixe en vert
  \draw[->, green, -Stealth] (D) -- (B);
  \draw[->, green, -Stealth] (B) -- (E);
  \draw[->, green, -Stealth] (E) -- (A);
  \draw[->, green, -Stealth] (A) -- (F);
  \draw[->, green, -Stealth] (F) -- (C);
  \draw[->, green, -Stealth] (C) -- (G);

    \node [right=of A, yshift=-1.5cm, xshift=3.5cm] {D$\rightarrow$B$\rightarrow$E$\rightarrow$A$\rightarrow$F$\rightarrow$C$\rightarrow$G};

\end{tikzpicture}

\paragraph{Parcours postfixe}
Parcours d'abord le sous-arbre gauche, puis le sous-arbre droit, et enfin visite la racine.
\newline
\begin{tikzpicture}[level distance=1.5cm,
  level 1/.style={sibling distance=3cm},
  level 2/.style={sibling distance=1.5cm}]
  \node (A) {A}
    child {node (B) {B}
      child {node (D) {D}}
      child {node (E) {E}}
    }
    child {node (C) {C}
      child {node (F) {F}}
      child {node (G) {G}}
    };

  % Parcours postfixe en bleu
  \draw[->, blue, -Stealth] (D) -- (E);
  \draw[->, blue, -Stealth] (E) -- (B);
  \draw[->, blue, -Stealth] (B) -- (F);
  \draw[->, blue, -Stealth] (F) -- (G);
  \draw[->, blue, -Stealth] (G) -- (C);
  \draw[->, blue, -Stealth] (C) -- (A);

\node [right=of A, yshift=-1.5cm, xshift=3.5cm] {D$\rightarrow$E$\rightarrow$B$\rightarrow$F$\rightarrow$G$\rightarrow$C$\rightarrow$A};
\end{tikzpicture}

\subsection{Arbres Binaires de Recherche (BST)}\label{subsec:arbres-binaires-de-recherche-(bst)}

\subsubsection{Définitions et Caractéristiques}
Un Arbre Binaire de Recherche (BST) est un arbre binaire dans lequel chaque nœud a une clé, et pour chaque nœud, toutes les clés dans le sous-arbre gauche sont inférieures à la clé du nœud, et toutes les clés dans le sous-arbre droit sont supérieures.
Cette propriété rend les BST efficaces pour la recherche et l'accès aux données.

\subsubsection{Opérations sur les BST}
Les opérations principales sur un BST incluent :
\begin{itemize}
    \item \textbf{Insertion} : Ajouter un nouvel élément tout en maintenant la propriété du BST.
    \item \textbf{Recherche} : Trouver un élément dans le BST.
    \item \textbf{Suppression} : Retirer un élément du BST.
    \item \textbf{Parcours} : Parcourir les éléments du BST (préfixe, infixe, postfixe).
\end{itemize}

\newpage
\subsubsection{Implémentation en C}
\begin{lstlisting}[label={lst:imp_BST}]
struct struct Node {
    int key;
    struct struct Node* left;
    struct struct Node* right;
};

struct Node* createNode(int key) {
    struct Node* newNode = (struct Node*) malloc(sizeof(struct Node));
    newNode->key = key;
    newNode->left = NULL;
    newNode->right = NULL;
    return newNode;
}
\end{lstlisting}
Cette structure et fonction en C représentent un nœud de BST et la création d'un nouveau nœud.

\subsubsection{Complexité des opérations}
\begin{tabular}{|l|c|c|}
\hline
 & \textbf{En moyenne} & \textbf{Dans le pire des cas} \\
\hline
\textbf{Insertion} & O(log n) & O(n) \\
\hline
\textbf{Recherche} & O(log n) & O(n) \\
\hline
\textbf{Suppression} & O(log n) & O(n) \\
\hline
\end{tabular}
La complexité dépend de l'équilibre de l'arbre.

\subsubsection{Exemples et Applications}
\begin{itemize}
    \item \textbf{Systèmes de gestion de bases de données} : Pour des recherches et des accès rapides aux données.
    \item \textbf{Algorithmes de tri} : Le tri par arbre est implémenté en utilisant des BST.
    \item \textbf{Structures de données dynamiques} : Les BST permettent l'insertion et la suppression de données tout en maintenant l'ordre.
\end{itemize}


\subsection{Tas}\label{subsec:tas}

\subsubsection{Définition et Types}
Un tas est une structure de données arborescente complète utilisée principalement pour implémenter des files de priorité.
Il existe deux types principaux de tas :
\begin{itemize}
    \item \textbf{Tas Max} : La clé à chaque nœud est supérieure aux clés de ses enfants.
    \item \textbf{Tas Min} : La clé à chaque nœud est inférieure aux clés de ses enfants.
\end{itemize}

\subsubsection{Représentation en C}
\begin{lstlisting}[label={lst:imp_heap}]
struct Heap {
    int* array;
    int size;
    int capacity;
};
\end{lstlisting}
Cette structure en C représente un tas avec un tableau pour stocker les éléments, la taille actuelle et la capacité maximale.

\subsubsection{Opérations sur les tas}
Les principales opérations sur les tas incluent :
\begin{itemize}
    \item \textbf{Insertion} : Ajouter un nouvel élément tout en maintenant la propriété du tas.
    \item \textbf{Suppression} : Retirer l'élément le plus élevé/bas (selon le type de tas).
    \item \textbf{Heapify} : Réorganiser le tas pour maintenir ses propriétés après une opération.
\end{itemize}

\subsubsection{Tri par tas}
Le tri par tas est un algorithme de tri efficace utilisant la structure de données du tas.
Il consiste à construire un tas à partir des données à trier, puis à déplacer successivement le plus grand élément du tas à la fin du tableau non trié.

\subsubsection{Applications des tas}
\begin{itemize}
    \item \textbf{Files de priorité} : Implémentation efficace pour gérer des éléments avec des priorités.
    \item \textbf{Tri de données} : Le tri par tas est un algorithme de tri en place avec une bonne complexité moyenne.
    \item \textbf{Algorithmes de graphe} : Comme dans l'algorithme de Dijkstra pour les chemins les plus courts.
\end{itemize}

\newpage